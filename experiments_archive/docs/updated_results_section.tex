\section{Experimental Results}
\label{sec:results}

We present comprehensive experimental results from our evaluation of topology-based privacy attacks across 1,136 unique configurations spanning both real-world deployments (5-30 nodes) and large-scale simulations (50-500 nodes). Our analysis examines attack effectiveness under varying privacy protections, identifies vulnerability patterns across network topologies, and quantifies the impact of privacy amplification mechanisms.

\subsection{Attack Performance Overview}

Our experimental evaluation reveals consistent vulnerabilities across all federated learning deployment scales. Communication pattern attacks demonstrate the highest effectiveness with an average success rate of 84.1\% (SD = 8.6\%) in real-world experiments, maintaining comparable performance in large-scale simulations. Parameter magnitude attacks exhibit moderate but stable performance at 65.0\% average success rate with notably low variance (SD = 3.9\%), indicating robust effectiveness across diverse network configurations. Topology structure attacks show the highest variability (mean = 47.2\%, SD = 22.4\%), reflecting their dependence on deployment-specific correlations between network position and data characteristics.

All tested attacks exceed the 30\% success threshold across all experimental conditions, confirming systematic vulnerabilities beyond random chance. Communication pattern attacks exhibit remarkable consistency across different experimental configurations, while topology structure attacks demonstrate the highest sensitivity to deployment-specific factors. Figure~\ref{fig:attack_effectiveness} illustrates attack success rates across different privacy protection levels, revealing fundamental limitations of current privacy protection mechanisms.

\begin{figure}[!t]
\centering
\includegraphics[width=0.5\textwidth]{figures/fig1_attack_effectiveness_heatmap.pdf}
\caption{Attack success rates across privacy protection levels. Communication pattern attacks achieve remarkable consistency with identical 84.1\% success rates across all differential privacy settings, demonstrating complete resistance to parameter-level privacy protection. This finding reveals a critical limitation of standard differential privacy mechanisms against metadata-based inference attacks.}
\label{fig:attack_effectiveness}
\end{figure}


\subsection{Privacy Protection Mechanism Performance}

Differential privacy mechanisms demonstrate critical limitations against topology-based attacks, revealing a fundamental gap in current privacy protection approaches. Across all tested per-round privacy protection levels ($\varepsilon \in \{1.0, 4.0, 8.0\}$), attack degradation remains below 25\%, with concerning variations in effectiveness across attack types.

\textbf{Communication Pattern Attack Resistance:} Most notably, communication pattern attacks exhibit complete resistance to differential privacy protection, maintaining identical 84.1\% success rates across all privacy levels from baseline to strong privacy ($\varepsilon = 1.0$ per round). This remarkable consistency reveals that parameter-level noise addition does not affect the underlying communication metadata that these attacks exploit. The attacks analyze timing patterns, message flow sequences, and communication frequencies—information that remains unchanged regardless of the Gaussian noise applied to model parameters.

\textbf{Limited Parameter Protection:} Parameter magnitude attacks show minimal sensitivity to privacy protection with only 2.3\% relative reduction (66.1\% to 64.6\%), maintaining success rates above 64\% under strong privacy constraints. This marginal improvement suggests that while differential privacy provides some protection against parameter-based inference, the protection is insufficient for practical security guarantees.

\textbf{Topology Structure Vulnerability:} Topology structure attacks demonstrate the strongest response to privacy protection, with a 17.6\% relative reduction in success rate (55.2\% to 45.5\%). However, even the most privacy-sensitive attacks maintain success rates above 45\% under the strongest tested privacy settings (per-round $\varepsilon = 1.0, \delta = 10^{-6}$).

These findings highlight a critical security gap: \textit{differential privacy mechanisms designed for parameter protection do not address metadata-based inference attacks}. The complete resistance of communication pattern attacks suggests that comprehensive privacy protection requires defense mechanisms beyond parameter-level noise addition.

\subsection{Network Scale Dependencies}

Attack effectiveness demonstrates remarkable stability across network sizes from 5 to 500 nodes. Figure~\ref{fig:network_scaling} combines real-world experiments with high-fidelity simulations to examine scaling behavior. Communication pattern attacks maintain success rates between 85.3\% (5 nodes) and 86.7\% (30 nodes) in real-world tests, with simulations showing comparable performance up to 500 nodes.

All tested topologies maintain attack detectability at large scales, with signal strength metrics remaining robust (0.68-0.99) across all network sizes. The gap between differential privacy-protected and unprotected scenarios remains consistent across all network sizes, indicating that privacy protection effectiveness does not improve with scale.

\begin{figure}[!t]
\centering
\subfloat[Attack effectiveness by topology]{
    \includegraphics[width=0.48\textwidth]{figures/fig5a_network_scaling.pdf}
    \label{fig:network_scaling_topology}
}
\hfill
\subfloat[Differential privacy impact]{
    \includegraphics[width=0.48\textwidth]{figures/fig5b_network_scaling.pdf}
    \label{fig:network_scaling_dp}
}
\caption{Network size scaling analysis showing attack success rates from 5 to 500 nodes. (a) Attack effectiveness across different topologies without differential privacy. (b) Impact of differential privacy on attack success. Real-world experiments (5-30 nodes, solid lines) show lower variance than simulations (50-500 nodes, dashed lines). Communication pattern attacks maintain effectiveness above 80\% across all scales.}
\label{fig:network_scaling}
\end{figure}

Real-world experiments (5-30 nodes) exhibit lower variance (SD = 3.9-8.6\%) than large-scale simulations (SD = 13.8-24.6\%). The increased variance in simulation results reflects our conservative modeling approach that introduces multiple variance sources including communication timing jitter, message size variations, and parameter update noise to prevent unrealistic homogeneity.

\subsection{Cross-Dataset Validation}

Attack effectiveness demonstrates remarkable consistency across different data domains. Figure~\ref{fig:dataset_violin} presents attack success distributions across MNIST and HAM10000 datasets, revealing substantial overlap (MNIST: 84.3\% average, HAM10000: 83.8\% average). The narrow confidence intervals (±2.1\% for MNIST, ±2.4\% for HAM10000) confirm statistical significance of cross-dataset consistency.

Medical imaging data exhibits equivalent vulnerability to simple digit classification tasks, with overlapping success rate distributions across all attack types. This domain-agnostic effectiveness indicates that data complexity and dimensionality do not provide inherent protection against topology-based inference.

\begin{figure}[!t]
\centering
\includegraphics[width=0.5\textwidth]{figures/fig4_dataset_violin.pdf}
\caption{Dataset vulnerability distributions showing overlapping profiles between MNIST and HAM10000. Medical imaging data exhibits equivalent vulnerability to simple digit classification, demonstrating domain-agnostic attack effectiveness.}
\label{fig:dataset_violin}
\end{figure}

\subsection{Privacy Amplification Through Subsampling}

Client and data subsampling exhibit limited effectiveness against topology-based attacks, with a counterintuitive non-monotonic reduction pattern. Figure~\ref{fig:subsampling_flow} illustrates privacy amplification results across different subsampling intensities. Under moderate subsampling conditions (50\% clients, 80\% data), average attack success decreases by only 6.4\% to 60.8\%. Strong subsampling (30\% clients, 60\% data) provides marginal improvement with a 3.6\% reduction to 62.6\%. Very strong subsampling (20\% clients, 50\% data) shows virtually no protection, reducing attack effectiveness by merely 0.2\% to 64.8\%.

The variance increases across all subsampling levels (standard deviation rising from 21.5\% to 26.8\%), indicating that sampling randomness introduces uncertainty without providing proportional privacy protection. Attack success rates remain well above security thresholds across all tested subsampling configurations.

\begin{figure}[!t]
\centering
\includegraphics[width=0.5\textwidth]{figures/fig3_subsampling_flow.pdf}
\caption{Privacy amplification effects through subsampling showing non-monotonic reduction pattern. Very strong subsampling conditions provide virtually no protection, demonstrating the inadequacy of traditional privacy amplification against topology-based attacks.}
\label{fig:subsampling_flow}
\end{figure}